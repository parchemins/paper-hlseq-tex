
\section{Introduction}

Distributed collaborative editors allow to distribute the work across space,
time and organizations. Some trending editors such as Google
Docs~\cite{nichols1995high} use the Operational Transform (OT)
approach~\cite{sun1998operational,sun1998achieving}. However, an alternative
based on Conflict-free Replicated Data Types
(CRDTs)~\cite{shapiro2011comprehensive,shapiro2011conflict} exists.  Compared
to OT, CRDTs are more decentralized and scale better.

The CRDTs belong to the optimistic
replication~\cite{saito2002replication,saito2005optimistic}
approach. Therefore, replicas involved in the collaboration are guaranteed
eventual convergence to an identical state. To provide convergence in a
replicated sequence, CRDTs use unique identifiers to link elements. When the
sequence is a document, the elements can be characters, lines, paragraphs, etc.


We distinguish two types of sequence CRDTs:
\begin{inparaenum}[(i)]
\item The tombstone
  CRDTs~\cite{ahmed2011evaluating,grishchenko2010deep,oster2006data,preguica2009commutative,roh2011replicated,weiss2007wooki,wu2010partial,Yu2012stringwise}
  use fixed-size identifiers. However, the delete operations only mark and hide
  elements to the users. Consequently, these deleted elements permanently
  affect performances.
\item The variable-size identifiers
  CRDTs~\cite{preguica2009commutative,weiss2009logoot} directly encode the
  order relation in the identifiers. However, their identifiers can grow
  linearly depending on the editing behaviour. Consequently, they remain unsafe
  in the distributed collaborative editing context.
\end{inparaenum}

To overcome their respective limitations, these approaches need an additional
protocol~\cite{letia2009crdts,roh2011replicated} related to garbage collection
mechanisms. Nevertheless, these protocols require global knowledge over
participants. In the general case, such knowledge remains prohibitively
expensive in the context of distributed networks subject to churn.

Recently, LSEQ~\cite{nedelec2013lseq} allowed variable-size identifiers CRDTs
to get rid of this costly additional protocol by lowering their upper bound on
space complexity from linear to sub-linear. Yet, LSEQ does not guarantee a safe
allocation. Indeed, it uses multiple antagonist strategies which, without any
coordination between collaborators, leads to an quadratic growth of
identifiers.

The contributions of this paper are:
\begin{inparaenum}[(i)]
\item experiments that highlight the negative effect of multiple users edition
  on the size of LSEQ identifiers
\item experiments that highlight that an increasing latency does not have a
  negative impact on the size of LSEQ identifiers.
\item an improvement of LSEQ called \NAME{} that extends the single user
  sub-linear upper bound on space complexity to any number of users, regardless
  of the latency, and without any additional cost.
\end{inparaenum} 

Ultimately, \NAME{} allows distributed collaborative editors to safely use
variable-size identifiers CRDTs.

The rest of the paper is organized as follow: Section~\ref{sec:background}
details variable-size identifiers CRDTs for sequences, the allocation strategy
LSEQ, and highlights the motivation of this paper.  Section~\ref{sec:proposal}
presents \NAME{} a combination of LSEQ and a hash-based choice strategy to
overcome the LSEQ limitations.  Section~\ref{sec:experiments} shows the results
of experiments over LSEQ and \NAME{} considering latency and multiple
users. Finally, Section~\ref{sec:relatedwork} reviews the related works.


%%% Local Variables: 
%%% mode: latex
%%% TeX-master: "../dchanges"
%%% End: 
