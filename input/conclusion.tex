
\section{Conclusion}

In this paper, we presented an improvement of the allocation strategy LSEQ
called \NAME{}. Contrarily to the original approach, our configuration supports
multiple users regardless of the latency. Indeed, replacing the random strategy
choice by the hash-based choice strategy allows to reach a global sub-linear
upper-bound on space complexity. As a consequence, distributed collaborative
editors can safely use \NAME{} and show better scalability than current
trending editors such as Google Docs, Etherpad, etc.

The hash-based choice strategy is a common function over participants that maps
a depth in the underlying tree to an allocation strategy identifier. This
surjective function initialized with a shared seed allows to reach an agreement
on strategies with no additional synchronization cost. Consequently, there is
no loss in the identifiers' space due to antagonist allocation strategies.

This paper highlights the importance of multiple users analysis in CRDTs for
sequences, particularly in the case of multiple underlying allocation
strategies. We also showed that latency does not badly affect the size of
identifiers in variable-size identifiers CRDTs. However, we performed
experiments with synthetic documents in order to show the general behaviour of
the allocation without considering semantically consistent documents.

Future works include the formal proof of the sub-linear upper-bound on space
complexity of \NAME{}. A probability analysis of the worst-case scenario is
mandatory to show that it seldom happens. We plan to experiment \NAME{} on a
corpus of real documents including multiple collaborators and
concurrency. Finally, we aim to build a real distributed collaborative editor
based on a variable-size identifiers CRDT using \NAME{} as allocation strategy.

%%% Local Variables: 
%%% mode: latex
%%% TeX-master: "../dchanges"
%%% End: 
